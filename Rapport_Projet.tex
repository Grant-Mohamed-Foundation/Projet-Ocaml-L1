\documentclass[a4paper, 11pt]{article}
\usepackage[utf8]{inputenc}
\usepackage[T1]{fontenc}
\usepackage[french]{babel}
\usepackage{graphicx}
\usepackage{amsmath}
\usepackage{amssymb}
\usepackage{hyperref}

\pagestyle{headings}

\title{Rapport de Projet 
\\ Programmation Fonctionnelle}
\date{}
\begin{document}
\begin{titlepage}
\center

{\Large \textsc{Université Jean Monnet}}

\vspace*{1cm}

{\Large \textsc{Faculté des Sciences et Techniques}}

\vfill

{\huge \bf \textsc{Projet de Programmation}}

\vspace*{1cm}

{\Large \textsc{Licence 1 Math-Info}}

%\vspace*{1cm}

{\large 2016-2017}

\vfill


\vspace*{0.5cm}

{\large \textsc{Binôme:}}

\vspace*{0.5cm}

Hugo Mohamed et Thomas Grant.

\end{titlepage}

\newpage
\tableofcontents
\newpage

\section{Fonctions de tri}
\subsection{Tri par sélection du minimum}
La fonction \texttt{tri\_selection\_min} utilise les fonctions \texttt{selectionne} et \texttt{supprime} pour selectionner le plus petit élément de la liste et de l'ajouter au début de la liste sans cet élément qui aura été supprimé de sa position initiale.

\subsection{Tri par partition-fusion}
On peut créer la fonction \texttt{tri\_partition\_fusion} en utilisant les fonctions \texttt{partitionne} et \texttt{fusionne} où la liste reçu est séparée en deux listes pour être triées séparément, puis elles sont fusionnées ensemble pour formée une liste triée suivant le comparateur.

\subsection{Tri par sélection du maximum}
Pour cette fonction de tri, nous avons fait une fonction bis pour pouvoir ajouté le plus grand élément de la liste que l'on veut traiter dans une nouvelle liste vide.

\subsection{Tri par insertion}
Pour cette fonction, on utilise une liste vide que l'on remplit en ajoutant les éléments de la liste donné au fur et à mesure dans le bon ordre.
\subsection{Tri par pivot}
Pour \texttt{tri\_pivot}, nous divisons la listes en 3 liste à partir du premier élément, puis nous recommençons jusqu'à obtenir pleins de listes à un élément que l'on concatène.

\subsection{Tri à bulle}
Cette fonction tri la liste en comparant les éléments deux à deux de manière récursive et en les intervertissant si nécessaire.
Au final, on compare notre liste initiale avec la liste modifiée et on recommence tant que les deux listes ne sont pas égales.

\newpage
\section{Choix d'une fonction de tri}
\subsection{Création d'un module Test efficacite}
Nous avons créé un module \texttt{efficacite} dans lequel nous avons définit \texttt{temps\_exe\_min} à partir de l'aide fourni dans le sujet pour connaître la vitesse d'un programme.\\
Nous pouvons donc ensuite comparer toutes nos fonctions de tri en fonction de la liste donné.
\subsection{Test}
Nous avons décidé de définir la fonction la plus rapide en se basant sur une moyenne des tests effectués avec des petites listes, puis des grandes listes, en utilisant la fonction \texttt{Random\_list}.\\En testant toutes les fonctions deux à deux, on se rend donc compte que la plus rapide est \texttt{tri\_pivot}.


\section{Module List sup définitif}
\subsection{Les Fonctions retenues}
La fonction \texttt{tri}:\\
\texttt{\#let tri = tri\_pivot} \\
On définit tout simplement la fonction la plus rapide comme étant la fonction de tri finale.\\ \\
La fonction \texttt{min\_list} qui permet d'avoir le minimum de la liste. \\ \\
Enfin \texttt{suppr\_doublons} qui renvoie notre liste sans doublons.
\end{document}
